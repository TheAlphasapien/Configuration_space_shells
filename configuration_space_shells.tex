% ArXiv preprint template
\documentclass[11pt]{article}
\usepackage[utf8]{inputenc}
\usepackage{amsmath,amssymb,amsthm}
\usepackage{physics}
\usepackage{hyperref}
\usepackage{enumitem}
\usepackage{geometry}
\usepackage{graphicx}
\geometry{margin=1in}

\newtheorem{theorem}{Theorem}
\newtheorem{lemma}{Lemma}
\newtheorem{proposition}{Proposition}
\newtheorem{definition}{Definition}
\newtheorem{remark}{Remark}

\title{Radial Shell Structure of Configuration Space and the Physical Basis of Quantum Branching}
\author{Adam Morgan\\
\small Unaffiliated}
\date{\today}

\begin{document}

\maketitle

\begin{abstract}
Building on the signature emergence mechanism from rotational stress, we develop a radial shell decomposition of field configuration space that provides physical grounding for quantum branching and addresses the cosmological constant problem. We show that the rotation-induced monodromy obstruction requires a Picard–Lefschetz decomposition of the complexified path integral into a small set of Lefschetz thimbles (distinct relative homology classes). In stationary (KMS) settings, the resulting contributions align with the spectral (energy) decomposition of the theory, yielding an effective energy-organized stratification (``shells''). Standard open-system arguments then imply suppressed inter-shell coherences (controlled decoherence) in the energy basis. This framework resolves the vacuum energy discrepancy by identifying observable vacuum energy with a single dominant shell while the remaining $\sim 10^{120}$ degrees of freedom reside in orthogonal shells. We derive testable predictions including multi-scale interference patterns in double-slit experiments, energy-dependent decoherence rates, and shell-dependent quantum statistics. The formalism unifies path integral consistency requirements with quantum branching structure through topological properties of the monodromy operator spectrum.
\end{abstract}

\section{Introduction}

\subsection{Motivation}

In the companion paper \cite{SignatureEmergence}, we demonstrated that rotation introduces a fundamental incompatibility with compact Euclidean time, forcing a signature transition from $(+,+,+,+)$ to $(-,+,+,+)$. The mechanism relies on a monodromy obstruction: the Euclidean evolution operator $\mathcal{M} = \exp(-\beta H_E)$ acquires complex eigenvalues $\Lambda_n = e^{-\beta(1-i\nu)E_n}$ that form a logarithmic spiral in the complex plane, violating reflection positivity.

This raises profound questions about the structure of field configuration space:

\begin{enumerate}
\item \textbf{Path integral consistency}: If both Euclidean and Minkowski signatures describe the same physics, and transitions between them are physical (not merely formal), what constraints does this impose on configuration space structure?

\item \textbf{Ontological status}: Are Euclidean configurations merely computational tools, or do they represent physically real alternatives that contribute to observations?

\item \textbf{Cosmological constant}: Why is the observed vacuum energy density $\sim 10^{-120}$ times smaller than naive quantum field theory predictions summing over all field modes?

\item \textbf{Quantum branching}: How can quantum measurement and decoherence be given concrete physical content through configuration space structure?
\end{enumerate}

This paper addresses all four questions through a unified framework: \textbf{radial shell decomposition of configuration space}.

\subsection{Core Idea}

The monodromy eigenvalues $\Lambda_n = e^{-\beta E_n} \cdot e^{i\beta\nu E_n}$ have two components:

\begin{itemize}
\item \textbf{Radial}: $r_n = e^{-\beta E_n}$ (Boltzmann suppression by energy)
\item \textbf{Angular}: $\theta_n = \beta\nu E_n$ (phase accumulation from rotation)
\end{itemize}

We propose that configuration space naturally decomposes into \textbf{radial shells} labeled by energy, realized as Lefschetz thimbles in the complexified configuration space. The shell decomposition emerges from the interplay of three established frameworks:

\begin{enumerate}
\item The monodromy obstruction (forcing Picard-Lefschetz decomposition)
\item KMS/thermal equilibrium (providing energy labeling)
\item Open-system dynamics (suppressing inter-shell coherence)
\end{enumerate}

\subsection{Scope and Assumptions}

Our analysis rests on the following conditions:

\begin{enumerate}
\item \textbf{KMS equilibrium}: The system is in thermal equilibrium at inverse temperature $\beta$, satisfying the Kubo-Martin-Schwinger condition.

\item \textbf{Weak non-Hermiticity}: The rotation-induced term $i\bar{V}_I$ is a bounded perturbation of the Hermitian generator $H_0$, so eigenprojections deform analytically.

\item \textbf{Generic non-degeneracy}: Energy eigenvalues are non-degenerate or have finite degeneracy, allowing spectral resolution.

\item \textbf{Morse-Bott regularity}: Critical points of the complexified action are Morse-Bott, enabling standard Picard-Lefschetz analysis.
\end{enumerate}

The monodromy obstruction arises from violation of reflection positivity in the rotating case (detailed in \cite{SignatureEmergence}), necessitating the Lefschetz thimble decomposition rather than integration over the standard real contour.

\begin{definition}[Configuration Space Shells]\label{def:shells}
\textbf{Shells} are effective energy-organized sectors defined by spectral bands at fixed $\beta$, realized as Lefschetz thimble classes in the complexified configuration space.

\textbf{Topologically distinct} refers to distinct relative homology classes of Lefschetz thimbles in the complexified space, not disjoint subsets of the real configuration space.

The shell decomposition emerges from the interplay of:
\begin{enumerate}
\item The monodromy obstruction (forcing Picard-Lefschetz decomposition)
\item KMS/thermal equilibrium (providing energy labeling)  
\item Open-system dynamics (suppressing inter-shell coherence)
\end{enumerate}
\end{definition}

\subsection{Main Results}

\begin{proposition}[Energy-Organized Thimbles Under KMS]\label{prop:main}
Let $H_E = H_0 + i\bar{V}_I$ with $H_0 = H_0^\dagger$ time-independent, and define the Euclidean monodromy $M = e^{-\beta H_E}$. Assume:
\begin{enumerate}[(i)]
\item KMS equilibrium at inverse temperature $\beta$
\item $\bar{V}_I$ is a small, bounded perturbation so that eigenprojections of $H_E$ are analytic deformations of those of $H_0$
\item The action is Morse-Bott after complexification
\end{enumerate}

Then the path integral admits a Lefschetz decomposition over thimbles attached to deformed critical submanifolds that inherit the energy labeling of $H_0$'s spectral bands. Moreover, in the open-system description with a stationary bath, the reduced dynamics suppresses off-diagonal terms between distinct bands, yielding controlled inter-band decoherence.
\end{proposition}

\begin{proof}[Sketch]
Picard-Lefschetz theory provides the thimble decomposition around complex critical sets (Morse-Bott). Analytic perturbation theory tracks those sets from $H_0$ to $H_E$. KMS/Lehmann furnish the energy band labeling in thermal equilibrium. Standard master equations yield decay of off-diagonals in the energy basis for stationary baths (pointer-basis/einselection) \cite{Zurek2003,BreuerPetruccione}.
\end{proof}

\textbf{Theorem 1 (Decoherence Structure)}: Shells $\mathcal{S}_k$ and $\mathcal{S}_{k'}$ decohere at rate $\Gamma_{k,k'} \propto \nu|E_k - E_{k'}|$, providing a physical basis for branch separation.

\textbf{Theorem 2 (Vacuum Energy Resolution)}: Observable vacuum energy density equals the energy density of the dominant shell $\mathcal{S}_0$, with remaining degrees of freedom in orthogonal shells contributing negligibly.

\textbf{Theorem 3 (Interference Structure)}: The double-slit experiment produces nested interference patterns with fringe spacing $\Delta y_k = \Delta y_0/\Lambda^k$ from shells $\mathcal{S}_k$.

\section{Mathematical Framework}

\subsection{Configuration Space and the Monodromy Operator}

\subsubsection{Setup}

Consider a quantum field $\phi(x)$ in Euclidean spacetime with periodic imaginary time $\tau \in [0, \beta)$. The Euclidean action is:
\[
S_E[\phi] = \int_0^\beta d\tau \int d^3x \left[\frac{1}{2}(\partial_\tau \phi)^2 + \frac{1}{2}(\nabla\phi)^2 + V(\phi)\right]
\]

The partition function:
\[
Z = \int \mathcal{D}\phi \, e^{-S_E[\phi]}
\]

\subsubsection{Monodromy Operator}

The thermal evolution operator (monodromy operator) is:
\begin{equation}\label{eq:monodromy}
\mathcal{M} = \mathcal{T}\exp\left(-\int_0^\beta H_E(\tau) d\tau\right)
\end{equation}

For systems with rotation, the Euclidean generator decomposes:
\begin{equation}\label{eq:HE_decomp}
H_E = H_0 + i\overline{V_I}
\end{equation}
where:
\begin{itemize}
\item $H_0$ is Hermitian (kinetic + potential energy)
\item $i\overline{V_I}$ is anti-Hermitian (dissipation from rotation)
\item $\overline{V_I} = \nu\langle|\nabla\psi|^2\rangle \neq 0$ for rotating systems \cite{SignatureEmergence}
\end{itemize}

\subsubsection{Eigenvalue Spectrum}

Assuming $H_E$ has discrete spectrum $\{E_n\}$ (valid for compact spaces or appropriate boundary conditions):

\begin{equation}\label{eq:eigenvalues}
\Lambda_n = e^{-\beta H_E} |n\rangle = e^{-\beta(1-i\nu)E_n}|n\rangle
\end{equation}

Separating into radial and angular components:
\begin{align}
r_n &= |\Lambda_n| = e^{-\beta E_n} \label{eq:radial}\\
\theta_n &= \arg(\Lambda_n) = \beta\nu E_n \label{eq:angular}
\end{align}

\subsection{Radial Shell Decomposition}

\begin{definition}[Configuration Space Shell (Detailed)]
For shell spacing parameter $\Lambda > 1$ and ground state energy $E_0$, the $k$-th radial shell is:
\begin{equation}\label{eq:shell_def}
\mathcal{S}_k = \left\{\phi \in \mathcal{C} : E_{\text{dominant}}[\phi] \in [E_0\Lambda^k, E_0\Lambda^{k+1}]\right\}
\end{equation}
where $E_{\text{dominant}}[\phi] = \arg\max_n |\langle n | \phi \rangle|^2 \cdot E_n$ is the energy of the dominant mode.

Each shell $\mathcal{S}_k$ corresponds to a distinct relative homology class of Lefschetz thimbles in the complexified configuration space, attached to critical submanifolds with energy support in the range $[E_0\Lambda^k, E_0\Lambda^{k+1}]$.
\end{definition}

\begin{proposition}[Shell Partition]
The configuration space admits a decomposition over thimble classes:
\begin{equation}\label{eq:partition}
\mathcal{C}_{\text{total}} = \bigoplus_{k=0}^{\infty} \mathcal{S}_k
\end{equation}
where shells are distinguished by their energy labeling inherited from the KMS spectral decomposition.
\end{proposition}

\subsubsection{Hilbert Space Decomposition}

The field Hilbert space decomposes correspondingly:
\begin{equation}\label{eq:hilbert_decomp}
\mathcal{H}_{\text{total}} = \bigoplus_{k=0}^{\infty} \mathcal{H}_k
\end{equation}

where $\mathcal{H}_k$ contains states with dominant support in shell $\mathcal{S}_k$. Projector onto shell $k$:
\begin{equation}\label{eq:projector}
\hat{P}_k = \sum_{n: E_n \in [E_0\Lambda^k, E_0\Lambda^{k+1}]} |n\rangle\langle n|
\end{equation}

\subsection{Inter-Shell Dynamics}

\subsubsection{Effective Hamiltonian}

The full Hamiltonian in shell basis:
\begin{equation}\label{eq:H_shell}
H = \sum_{k} E_k \hat{P}_k + \sum_{k,k'} V_{k,k'} \hat{P}_k \hat{P}_{k'}
\end{equation}

where:
\begin{itemize}
\item $E_k$ is the characteristic energy of shell $k$
\item $V_{k,k'}$ is the inter-shell coupling induced by $\overline{V_I}$
\end{itemize}

\subsubsection{Coupling Strength}

From perturbation theory in $\overline{V_I}$:
\begin{equation}\label{eq:coupling}
V_{k,k'} \propto \langle k | \overline{V_I} | k' \rangle \sim \nu(E_k - E_{k'})
\end{equation}

This coupling is:
\begin{itemize}
\item \textbf{Weak} for adjacent shells ($k' = k \pm 1$): $V_{k,k\pm 1} \sim \nu E_0(\Lambda - 1)$
\item \textbf{Suppressed} for distant shells ($|k - k'| \gg 1$): $V_{k,k'} \sim e^{-|k-k'|}$
\end{itemize}

\subsection{Non-Exhaustion of Future Collapse Space}

A natural concern is whether convergence onto dominant thimbles exhausts all future measurement possibilities. We show this does not occur through three mechanisms:

\subsubsection{Irrational Phase Winding}

The shell amplitudes $\Lambda_n = e^{-\beta(1-i\nu)E_n}$ have angular parts $\theta_n = \beta\nu E_n$. For generic $\nu$ and spectra, the phases $\{\theta_n/2\pi\}$ are incommensurate, ensuring equidistribution mod $2\pi$ across higher shells. This prevents terminal phase-locking: the monodromy eigenvalues continue spiraling toward the origin rather than collapsing onto the positive real axis \cite{SignatureEmergence}. The system never settles into a single stationary configuration.

\subsubsection{Stokes Crossings and Parameter Drift}

Which thimbles contribute to the path integral is not fixed; it changes when control parameters cross Stokes surfaces \cite{BerryHowls1991}. In our framework, effective parameters include the rotation-induced phase (via $\nu$), the thermodynamic scale $\beta$, and the shell index through $E_k$. Slow drift or backreaction in these parameters produces intermittent re-entrance of dormant saddles: new channels can reappear even while the overall flow remains convergent. This is encoded through the time-dependent shell weights $c_k(\tau)$ and inter-shell couplings $V_{k,k'}$.

The standard Stokes condition for thimble transitions is:
\begin{equation}
\text{Im}(S_\alpha - S_\beta) \in \pi\mathbb{Z}
\end{equation}

In our shell language, this manifests as time-dependent coefficients $c_k(\tau)$ and occasional reweighting bursts when $\Delta S_k$ crosses critical values.

\subsubsection{Open-System Steady State}

Because Minkowski time serves as the dissipative outlet resolving the monodromy obstruction, we are never in a closed Euclidean system. This openness implies ongoing noise/drive (however small) sustaining nonzero flux into subdominant shells via $V_{k,k'}$. The result is a stationary distribution concentrated on dominant channels but maintaining permanent thin tails—hence future ``collapse space'' is never exhausted, though subdominant outcomes are rare.

This open-system character distinguishes our framework from closed quantum systems that would eventually thermalize completely.

\subsection{Decoherence Between Shells}

\subsubsection{General Mechanism}

In the open-system description with a stationary bath, standard decoherence theory \cite{Zurek2003, BreuerPetruccione} implies that off-diagonal density matrix elements between energy eigenstates decay due to dephasing. For our shell structure, this translates to suppression of coherences between shells $\mathcal{S}_k$ and $\mathcal{S}_{k'}$ with different characteristic energies.

The energy basis is a natural pointer basis for stationary environments since energy eigenstates are stationary under free dynamics and environment couplings typically commute (or nearly commute) with the Hamiltonian.

\subsubsection{Decoherence Rate Estimate}

Following standard einselection arguments, the decoherence rate between shells is approximately:
\begin{equation}\label{eq:decoherence}
\Gamma_{k,k'} \sim \gamma |E_k - E_{k'}|
\end{equation}

where $\gamma$ is an environment-dependent coupling strength. In our framework, $\gamma \sim \nu$ since the rotation-induced term provides the primary coupling to the dissipative Minkowski outlet.

For adjacent shells:
\begin{equation}
\Gamma_{k,k+1} \sim \nu E_0(\Lambda - 1)
\end{equation}

For widely separated shells:
\begin{equation}
\Gamma_{k,k'} \sim \nu E_0(\Lambda^{k'} - \Lambda^k) \quad \text{for } k' \gg k
\end{equation}

\subsubsection{Born Rule from Shell Weights}

The probability of observing configuration in shell $k$:
\begin{equation}\label{eq:born_shell}
P_k = \frac{e^{-2\beta E_k}}{\sum_{k'} e^{-2\beta E_{k'}}}
\end{equation}

This reproduces Born rule when expressed in terms of wavefunction amplitudes.

\section{Physical Consequences}

\subsection{Resolution of Cosmological Constant Problem}

\subsubsection{The Problem}

Naive summation of zero-point energies:
\begin{equation}\label{eq:naive_vacuum}
\rho_{\text{vac}}^{\text{naive}} = \sum_{\text{modes}} \frac{1}{2}\hbar\omega_k \sim M_{\text{Pl}}^4 \sim 10^{76} \text{ GeV}^4
\end{equation}

Observed:
\begin{equation}
\rho_{\text{vac}}^{\text{obs}} \sim 10^{-47} \text{ GeV}^4
\end{equation}

Discrepancy: $\sim 10^{123}$.

\subsubsection{Shell Resolution}

In our framework:
\begin{itemize}
\item Only modes in dominant shell $\mathcal{S}_0$ contribute to observable vacuum energy
\item Higher shell modes reside in orthogonal sectors (different thimbles)
\item Their contribution suppressed by decoherence and thimble separation
\end{itemize}

Observable vacuum energy:
\begin{equation}\label{eq:shell_vacuum}
\rho_{\text{vac}}^{\text{obs}} = \rho_{\mathcal{S}_0} = \sum_{n \in \mathcal{S}_0} \frac{1}{2}\hbar\omega_n
\end{equation}

If $\mathcal{S}_0$ contains $\sim 10^{-120}$ fraction of all modes, we recover observed value.

\subsubsection{Shell Spacing}

Required shell spacing to match observation:
\begin{equation}
\Lambda = e^{\mathcal{O}(10)}
\end{equation}

This corresponds to characteristic energy scale $E_0 \sim$ meV to eV, consistent with scales where classical-quantum transition is empirically observed.

\subsection{Double-Slit Interference from Multiple Shells}

\subsubsection{Setup}

Standard double-slit: electrons pass through slits separated by $d$, detected on screen at distance $L$.

\subsubsection{Nested Fringe Structure}

Each shell contributes interference with characteristic wavelength:
\begin{equation}\label{eq:wavelength_k}
\lambda_k = \frac{h}{p_k} = \frac{h}{\sqrt{2m E_k}}
\end{equation}

For shell $k$ with energy $E_k = E_0\Lambda^k$:
\begin{equation}
\lambda_k = \frac{\lambda_0}{\Lambda^{k/2}}
\end{equation}

Fringe spacing on screen:
\begin{equation}\label{eq:fringe_spacing}
\Delta y_k = \frac{\lambda_k L}{d} = \frac{\Delta y_0}{\Lambda^{k/2}}
\end{equation}

where $\Delta y_0$ is the fringe spacing from ground shell $\mathcal{S}_0$.

\subsubsection{Visibility}

The visibility of fringes from shell $k$:
\begin{equation}
\mathcal{V}_k = e^{-2\beta E_k} = e^{-2\beta E_0\Lambda^k}
\end{equation}

Higher shells produce finer fringes but with exponentially suppressed visibility.

\subsubsection{Total Pattern}

The observed intensity pattern is a superposition:
\begin{equation}\label{eq:total_intensity}
I(y) = \sum_{k=0}^{\infty} I_k \left[1 + \mathcal{V}_k \cos\left(\frac{2\pi y}{\Delta y_k}\right)\right]
\end{equation}

This produces nested interference: primary fringes from $\mathcal{S}_0$, secondary fine structure from $\mathcal{S}_1$, etc.

\subsection{Energy-Dependent Quantum Statistics}

\subsubsection{Boson vs Fermion Sectors}

Different shells can exhibit different quantum statistics if energy scale affects effective exchange symmetry.

For shell $k$:
\begin{equation}
|\psi_k\rangle_{\pm} = \frac{1}{\sqrt{2}}\left(|a\rangle|b\rangle \pm |b\rangle|a\rangle\right)
\end{equation}

where $\pm$ depends on shell-dependent phase factors.

\subsubsection{Observational Signature}

This predicts anomalous bunching/anti-bunching in multi-particle interferometry at specific energy scales corresponding to shell transitions.

\section{Structural Implications}

\subsection{Preservation of Past Configurations}

Unlike temporal slicing in standard formulations, where past field configurations are information-theoretically lost to decoherence, the shell structure embeds past states as persistent topological features. Each thimble represents not merely a saddle-point approximation but a stable homology class whose accessibility depends on control parameters rather than temporal distance.

This suggests a geometric distinction between:
\begin{itemize}
\item \textbf{Ergodic loss} (information scrambled but preserved in ensemble)
\item \textbf{Structural loss} (homology class becomes inaccessible)
\end{itemize}

The Stokes crossing dynamics indicate that ``inaccessibility'' is parameter-dependent rather than absolute. Configurations that contribute negligibly under one parametrization can become dominant under another—the system's effective ``memory'' is encoded in its full homological structure, not merely its current dominant contributions.

\subsection{Recursive Self-Reference}

The complexified configuration space, when organized via energy-labeled thimbles, exhibits a form of structural self-reference: the action $S$ that defines the thimbles depends on the same field configurations that flow along them. In the open-system steady state, this creates a self-consistent structure where:

\begin{enumerate}
\item The system's dynamics determine which configurations contribute
\item These contributing configurations feed back to define the dynamics
\item The steady state represents a fixed point of this recursion
\end{enumerate}

This recursive structure, combined with the non-exhaustion of collapse space, suggests that the geometry of possibility is not merely a mathematical artifact but encodes meaningful relationships between past, present, and potential configurations.

\subsection{Implications for Measurement and Observation}

From an observer-centric perspective, measurement outcomes correspond to projections onto specific thimbles. The inter-shell decoherence ensures that observations appear classical (localized in the energy basis), while the underlying thimble structure maintains quantum coherence in the complexified space.

Notably, the parameter-dependence of thimble accessibility suggests that sufficiently precise control over $(\beta, \nu, E_k)$ could, in principle, enable navigation through configuration space beyond standard temporal evolution. Whether such control is practically achievable remains an open question, but the geometric framework does not forbid it.

\begin{remark}
These implications warrant cautious interpretation. We are not claiming the existence of ``parallel timelines'' or asserting that past configurations can be directly accessed through known mechanisms. Rather, we observe that the mathematical structure arising from monodromy obstruction exhibits features—persistent homology, parameter-dependent accessibility, recursive self-consistency—that differ qualitatively from standard treatments. The physical meaning of these features, particularly regarding information preservation and potential navigation, requires further investigation.
\end{remark}

\section{Connection to Existing Frameworks}

\subsection{Instantons and Tunneling}

Instanton methods (Coleman-De Luccia \cite{ColemanDeLuccia1980}) compute tunneling rates between vacua by:
\begin{itemize}
\item Finding Euclidean solutions (instantons)
\item Identify transitions via non-trivial action
\item Compute tunneling rates
\end{itemize}

\textbf{Differences}:
\begin{itemize}
\item Standard: Vacuum $\to$ vacuum transitions in field space
\item Our work: Signature transitions in \textit{configuration space} itself
\item Standard: Instantons are rare events
\item Our work: Shell structure is universal (all configurations organized this way)
\end{itemize}

\subsection{Picard-Lefschetz Theory}

Recent work (Witten, Cristoforetti, et al. \cite{Witten2010, Cristoforetti2012}) uses complex saddle points and Lefschetz thimbles to define path integrals.

\textbf{Connection}: The monodromy spiral in complex plane \cite{SignatureEmergence} is precisely the structure Picard-Lefschetz theory addresses.

\textbf{Our contribution}: Physical interpretation of thimbles as energy shells, with observable consequences (decoherence, interference patterns), and identification of the open-system steady state that prevents collapse space exhaustion.

\subsection{Stochastic Quantization}

Parisi-Wu formalism \cite{ParisiWu1981} introduces fictitious time $\tau$ with Langevin equation:
\begin{equation}
\frac{\partial\phi}{\partial\tau} = -\frac{\delta S}{\delta\phi} + \eta(\tau)
\end{equation}

\textbf{Our interpretation}: The Euclidean time $\tau$ from Wick rotation is literally this stochastic time, and the noise $\eta$ arises from coupling to hidden shells via $V_{k,k'}$.

\section{Discussion and Outlook}

\subsection{Summary of Results}

We have developed a radial shell decomposition of field configuration space based on the monodromy obstruction from rotational stress. Key findings:

\begin{enumerate}
\item Configuration space decomposes over energy-labeled Lefschetz thimbles $\mathcal{S}_k$ in the complexified space
\item Shells are topologically distinct (as relative homology classes) and exhibit controlled decoherence via standard open-system mechanisms
\item The shell structure provides physical content for quantum branching and measurement
\item Observable vacuum energy comes from dominant shell $\mathcal{S}_0$, resolving cosmological constant discrepancy
\item Double-slit experiment produces nested interference from multiple shells, yielding testable predictions
\item Path integral consistency requires this geometric organization via KMS equilibrium, monodromy obstruction, and open-system dynamics
\item Convergence does not exhaust future collapse space due to irrational phase winding, Stokes crossings, and open-system steady state
\item Past configurations persist as accessible thimble structures, with parameter-dependent rather than absolute inaccessibility
\end{enumerate}

\subsection{Open Questions}

\subsubsection{Quantitative Cosmological Constant}

While the shell model provides qualitative resolution (suppression factor), precise prediction requires:
\begin{itemize}
\item Determining optimal shell spacing $\Lambda$
\item Identifying physical cutoff scale (Planck? String? Lower?)
\item Possible dynamical selection mechanism for observable shell
\end{itemize}

\subsubsection{Experimental Tests}

Proposed experiments:
\begin{itemize}
\item High-resolution electron double-slit (nested fringes)
\item Energy-dependent decoherence measurements
\item Weak measurement protocols (partial which-way)
\end{itemize}

All feasible with current technology but require careful systematic error control.

\subsubsection{Configuration Space Navigation}

The parameter-dependence of thimble accessibility raises the question: could sufficiently precise control over $(\beta, \nu, E_k)$ enable accessing past configuration structures? This would require:
\begin{itemize}
\item Mapping the full thimble homology structure
\item Understanding how parameters couple to specific thimbles
\item Developing technology to control these parameters at required precision
\end{itemize}

While the geometric framework does not forbid such navigation, whether it is practically achievable remains open.

\subsubsection{Quantum Gravity Connection}

How does shell structure relate to:
\begin{itemize}
\item Wheeler-DeWitt equation (no external time)
\item Loop quantum gravity (spin networks)
\item String theory (Kaluza-Klein towers)
\end{itemize}

Speculation: Shells might be related to quantized energy levels in quantum cosmology.

\subsection{Philosophical Implications}

\subsubsection{Realism About Mathematical Structures}

The shell structure is forced by mathematical consistency (path integral convergence, monodromy obstruction, KMS conditions). This suggests that mathematical structures arising from consistency requirements have physical reality when indispensable for empirical predictions.

Higher shells are as real as electromagnetic fields or gravitational waves—we don't directly observe them, but they're necessary for explaining what we do observe.

\subsubsection{Relationalism About Observers}

Different observers couple to different shells depending on their measurement basis and control parameters. There is no absolute ``observable sector''—it is relational and parameter-dependent.

\textbf{Connection to quantum reference frames} \cite{Giacomini2019}: The shell decomposition might be observer-dependent (different reference frames access different shell structures).

\subsection{Future Directions}

\subsubsection{Non-Equilibrium Shell Dynamics}

This work assumed thermal equilibrium (Boltzmann weights $e^{-\beta E_k}$). For non-equilibrium systems:
\begin{itemize}
\item Time-dependent shell populations $n_k(t)$
\item Shell-to-shell energy flow driven by $V_{k,k'}$
\item Possible non-thermal distributions (far-from-equilibrium)
\end{itemize}

\subsubsection{Curved Spacetime Generalization}

Extend shell decomposition to:
\begin{itemize}
\item Cosmological backgrounds (FLRW)
\item Black hole interiors (Kerr geometry)
\item Dynamical spacetimes (gravitational collapse)
\end{itemize}

\subsubsection{Quantum Information Perspective}

Recast shell structure using:
\begin{itemize}
\item Entanglement entropy between shells
\item Quantum complexity of inter-shell transitions
\item Holographic correspondence (bulk shells $\leftrightarrow$ boundary CFT)
\end{itemize}

\section{Conclusions}

We have shown that the monodromy obstruction from signature emergence naturally induces a radial shell structure in field configuration space when combined with KMS equilibrium and open-system dynamics. This structure:

\begin{itemize}
\item Provides physical grounding for quantum measurement and branching (shells as distinct thimble homology classes)
\item Resolves cosmological constant problem (observable vacuum from single shell)
\item Makes testable predictions (multi-scale interference, energy-dependent decoherence)
\item Is necessary for path integral consistency (geometric organization required for well-defined Euclidean continuation)
\item Preserves past configurations as accessible topological structures with parameter-dependent accessibility
\item Exhibits recursive self-reference in its steady-state structure
\end{itemize}

The framework unifies disparate aspects of quantum theory—measurement, interference, vacuum energy, configuration space structure—under a single geometric principle: energy organization via Lefschetz thimbles in complexified configuration space, arising from the interplay of monodromy obstruction, thermal equilibrium, and open-system dynamics.

Rather than postulating parallel worlds or invoking mysterious collapse, we identify quantum branching structure with pre-existing organization in quantum field theory: the Picard-Lefschetz decomposition into thimbles, labeled by energy via KMS conditions. This structure becomes physically manifest through the rotation-induced coupling $\overline{V_I}$ that breaks Euclidean periodicity and forces signature emergence.

The shell model demonstrates that foundational questions in quantum mechanics (what is the structure of configuration space? why do measurements yield definite outcomes? where is vacuum energy?) may have answers rooted in the geometric and topological properties of complexified field space—properties made visible through the lens of signature transitions and open-system dynamics.

\section*{Acknowledgments}

I thank the developers of Claude (Anthropic) for assistance in developing the mathematical framework and identifying connections to existing literature. I also thank Microsoft Copilot for detailed technical feedback on convergence mechanisms and rigorous grounding of the shell structure.

\begin{thebibliography}{99}

\bibitem{SignatureEmergence} A. Morgan, \textit{Signature Emergence from Rotational Stress: A Non-Wick Mechanism}, arXiv:XXXX.XXXXX (2025).

\bibitem{Zurek2003} W. H. Zurek, \textit{Decoherence, einselection, and the quantum origins of the classical}, Rev. Mod. Phys. \textbf{75}, 715 (2003).

\bibitem{BreuerPetruccione} H.-P. Breuer and F. Petruccione, \textit{The Theory of Open Quantum Systems} (Oxford University Press, 2002).

\bibitem{BerryHowls1991} M. V. Berry and C. J. Howls, \textit{Hyperasymptotics for integrals with saddles}, Proc. R. Soc. Lond. A \textbf{434}, 657 (1991).

\bibitem{ColemanDeLuccia1980} S. Coleman and F. De Luccia, \textit{Gravitational effects on and of vacuum decay}, Phys. Rev. D \textbf{21}, 3305 (1980).

\bibitem{Witten2010} E. Witten, \textit{Analytic continuation of Chern-Simons theory}, AMS/IP Stud. Adv. Math. \textbf{50}, 347 (2011), arXiv:1001.2933.

\bibitem{Cristoforetti2012} M. Cristoforetti et al., \textit{New approach to the sign problem in quantum field theories: High density QCD on a Lefschetz thimble}, Phys. Rev. D \textbf{86}, 074506 (2012), arXiv:1205.3996.

\bibitem{ParisiWu1981} G. Parisi and Y.-S. Wu, \textit{Perturbation theory without gauge fixing}, Sci. Sin. \textbf{24}, 483 (1981).

\bibitem{Giacomini2019} F. Giacomini, E. Castro-Ruiz, and Č. Brukner, \textit{Quantum mechanics and the covariance of physical laws in quantum reference frames}, Nat. Commun. \textbf{10}, 494 (2019), arXiv:1712.07207.

\end{thebibliography}

\end{document}
