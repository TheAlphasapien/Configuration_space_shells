% ArXiv preprint template
\documentclass[11pt]{article}
\usepackage[utf8]{inputenc}
\usepackage{amsmath,amssymb,amsthm}
\usepackage{physics}
\usepackage{hyperref}
\usepackage{enumitem}
\usepackage{geometry}
\usepackage{graphicx}
\geometry{margin=1in}

\newtheorem{theorem}{Theorem}
\newtheorem{lemma}{Lemma}
\newtheorem{proposition}{Proposition}
\newtheorem{definition}{Definition}
\newtheorem{remark}{Remark}

\title{Radial Shell Structure of Configuration Space and the Physical Basis of Quantum Convergence}
\author{Adam Morgan\\
\small Unaffiliated}
\date{\today}

\begin{document}

\maketitle

\begin{abstract}
Building on the signature emergence mechanism from rotational stress, we demonstrate that field configuration space exhibits a convergent structure rather than unbounded branching. The rotation-induced monodromy obstruction at black hole boundaries requires organizing boundary fluid configurations via Picard–Lefschetz thimbles—distinct relative homology classes in complexified configuration space. Each boundary point defines a valid Minkowski reference frame, yet the path integral converges onto dominant configurations rather than proliferating into separate worlds. In stationary (KMS) settings, thimbles align with spectral (energy) decomposition, yielding energy-organized shells with controlled decoherence that suppress subdominant contributions. This convergent structure resolves three problems: (1) it explains why we observe a single classical reality despite quantum superposition existing at every boundary point, (2) it accounts for vacuum energy discrepancy by confining observable contributions to the dominant shell while $\sim 10^{120}$ degrees of freedom reside in geometrically persistent but observationally suppressed configurations, and (3) it preserves past states as accessible thimble structure rather than information-theoretically erasing them. We derive testable predictions including multi-scale interference patterns, demonstrate that Stokes crossing dynamics enable parameter-dependent access to non-dominant configurations, and show how the formalism provides physical grounding for the convergent nature of quantum measurement.
\end{abstract}

\section{Introduction}

\subsection{Motivation}

In the companion paper \cite{SignatureEmergence}, we demonstrated that rotation introduces a fundamental incompatibility with compact Euclidean time, forcing a signature transition from $(+,+,+,+)$ to $(-,+,+,+)$. The mechanism relies on a monodromy obstruction: the Euclidean evolution operator $\mathcal{M} = \exp(-\beta H_E)$ acquires complex eigenvalues $\Lambda_n = e^{-\beta(1-i\nu)E_n}$ that form a logarithmic spiral in the complex plane, violating reflection positivity.

This raises profound questions about the structure of field configuration space:

\begin{enumerate}
\item \textbf{Path integral consistency}: If both Euclidean and Minkowski signatures describe the same physics, and transitions between them are physical (not merely formal), what constraints does this impose on configuration space structure?

\item \textbf{Ontological status}: Are Euclidean configurations merely computational tools, or do they represent physically real alternatives that contribute to observations?

\item \textbf{Cosmological constant}: Why is the observed vacuum energy density $\sim 10^{-120}$ times smaller than naive quantum field theory predictions summing over all field modes?

\item \textbf{Quantum convergence}: How can quantum measurement and decoherence be given concrete physical content through configuration space structure?
\end{enumerate}

This paper addresses all four questions through a unified framework: \textbf{radial shell decomposition of configuration space}.

\subsection{Core Idea}

The monodromy eigenvalues $\Lambda_n = e^{-\beta E_n} \cdot e^{i\beta\nu E_n}$ have two components:

\begin{itemize}
\item \textbf{Radial}: $r_n = e^{-\beta E_n}$ (Boltzmann suppression by energy)
\item \textbf{Angular}: $\theta_n = \beta\nu E_n$ (phase accumulation from rotation)
\end{itemize}

We propose that configuration space naturally decomposes into \textbf{radial shells} labeled by energy, realized as Lefschetz thimbles in the complexified configuration space. The shell decomposition emerges from the interplay of three established frameworks:

\begin{enumerate}
\item The monodromy obstruction (forcing Picard-Lefschetz decomposition)
\item KMS/thermal equilibrium (providing energy labeling)
\item Open-system dynamics (driving convergence onto dominant shells)
\end{enumerate}

\subsection{Scope and Assumptions}

Our analysis rests on the following conditions:

\begin{enumerate}
\item \textbf{KMS equilibrium}: The system is in thermal equilibrium at inverse temperature $\beta$, satisfying the Kubo-Martin-Schwinger condition.

\item \textbf{Weak non-Hermiticity}: The rotation-induced term $i\bar{V}_I$ is a bounded perturbation of the Hermitian generator $H_0$, so eigenprojections deform analytically.

\item \textbf{Generic non-degeneracy}: Energy eigenvalues are non-degenerate or have finite degeneracy, allowing spectral resolution.

\item \textbf{Morse-Bott regularity}: Critical points of the complexified action are Morse-Bott, enabling standard Picard-Lefschetz analysis.
\end{enumerate}

The monodromy obstruction arises from violation of reflection positivity in the rotating case (detailed in \cite{SignatureEmergence}), necessitating the Lefschetz thimble decomposition rather than integration over the standard real contour.

\begin{definition}[Configuration Space Shells]\label{def:shells}
\textbf{Shells} are effective energy-organized sectors defined by spectral bands at fixed $\beta$, realized as Lefschetz thimble classes in the complexified configuration space.

\textbf{Topologically distinct} refers to distinct relative homology classes of Lefschetz thimbles in the complexified space, not disjoint subsets of the real configuration space.

The shell decomposition emerges from the interplay of:
\begin{enumerate}
\item The monodromy obstruction (forcing Picard-Lefschetz decomposition)
\item KMS/thermal equilibrium (providing energy labeling) 
\item Open-system dynamics (driving convergence onto dominant shells)
\end{enumerate}
\end{definition}

\subsection{Main Results}

\begin{proposition}[Energy-Organized Thimbles Under KMS]
Let $H_E = H_0 + i\bar{V}_I$ with $H_0 = H_0^\dagger$ time-independent, and define the Euclidean monodromy $M = e^{-\beta H_E}$. Under the assumptions stated in Section 1.3 (KMS equilibrium, weak non-Hermiticity, generic non-degeneracy, and Morse-Bott regularity), the path integral admits a Lefschetz decomposition over thimbles attached to deformed critical submanifolds that inherit the energy labeling of $H_0$'s spectral bands. Moreover, in the open-system description with a stationary bath, the reduced dynamics suppresses off-diagonal terms between distinct bands, yielding controlled inter-band decoherence.
\end{proposition}

\begin{proof}[Sketch]
Picard-Lefschetz theory provides the thimble decomposition around complex critical sets (Morse-Bott). Analytic perturbation theory tracks those sets from $H_0$ to $H_E$. KMS/Lehmann furnish the energy band labeling in thermal equilibrium. Standard master equations yield decay of off-diagonals in the energy basis for stationary baths (pointer-basis/einselection) \cite{Zurek2003,BreuerPetruccione}.
\end{proof}

\textbf{Theorem 1 (Convergence Structure)}: Shells $\mathcal{S}_k$ and $\mathcal{S}_{k'}$ decohere at rate $\Gamma_{k,k'} \propto \nu|E_k - E_{k'}|$, suppressing coherence between non-dominant configurations and driving convergence onto the dominant shell.

\textbf{Theorem 2 (Vacuum Energy Resolution)}: Observable vacuum energy density equals the energy density of the dominant shell $\mathcal{S}_0$, with remaining degrees of freedom in orthogonal shells contributing negligibly.

\textbf{Theorem 3 (Interference Structure)}: The double-slit experiment produces nested interference patterns with fringe spacing $\Delta y_k = \Delta y_0/\Lambda^k$ from shells $\mathcal{S}_k$.

\section{Mathematical Framework}



\subsection{Configuration Space and the Monodromy Operator}

\subsubsection{Setup}

Consider a quantum field $\phi(x)$ in Euclidean spacetime with periodic imaginary time $\tau \in [0, \beta)$. The Euclidean action is:
\[
S_E[\phi] = \int_0^\beta d\tau \int d^3x \left[\frac{1}{2}(\partial_\tau \phi)^2 + \frac{1}{2}(\nabla\phi)^2 + V(\phi)\right]
\]

The partition function:
\[
Z = \int \mathcal{D}\phi \, e^{-S_E[\phi]}
\]

\subsubsection{Monodromy Operator}

The thermal evolution operator (monodromy operator) is:
\begin{equation}\label{eq:monodromy}
\mathcal{M} = \mathcal{T}\exp\left(-\int_0^\beta H_E(\tau) d\tau\right)
\end{equation}

For systems with rotation, the Euclidean generator decomposes:
\begin{equation}\label{eq:HE_decomp}
H_E = H_0 + i\overline{V_I}
\end{equation}
where:
\begin{itemize}
\item $H_0$ is Hermitian (kinetic + potential energy)
\item $i\overline{V_I}$ is anti-Hermitian (dissipation from rotation)
\item $\overline{V_I} = \nu\langle|\nabla\psi|^2\rangle \neq 0$ for rotating systems \cite{SignatureEmergence}
\end{itemize}

\subsubsection{Eigenvalue Spectrum}

Assuming $H_E$ has discrete spectrum $\{E_n\}$ (valid for compact spaces or appropriate boundary conditions):

\begin{equation}\label{eq:eigenvalues}
\Lambda_n = e^{-\beta H_E} |n\rangle = e^{-\beta(1-i\nu)E_n}|n\rangle
\end{equation}

Separating into radial and angular components:
\begin{align}
r_n &= |\Lambda_n| = e^{-\beta E_n} \label{eq:radial}\\
\theta_n &= \arg(\Lambda_n) = \beta\nu E_n \label{eq:angular}
\end{align}

\subsection{Physical Interpretation: Thimbles as Boundary States}

The Lefschetz thimbles arising from monodromy obstruction are not merely 
mathematical artifacts but have direct physical interpretation as boundary 
fluid configurations.

\subsubsection{From Black Hole Boundaries to Configuration Space}

In the companion paper \cite{SignatureEmergence}, we demonstrated that 
rotating systems exhibit monodromy obstruction at their boundaries. For 
black holes, the stretched horizon acts as a boundary fluid with:

\begin{itemize}
\item \textbf{Dissipative dynamics}: Viscosity parameter $\nu$
\item \textbf{Rotational stress}: Vorticity $\omega$ from frame-dragging 
\item \textbf{Thermal character}: Temperature $T_H = 1/\beta = \kappa/(4\pi)$
\end{itemize}

The Euclidean path integral over boundary states requires complexification 
due to the monodromy obstruction. The resulting Picard-Lefschetz 
decomposition expresses the partition function as:

$$Z = \sum_{\text{thimbles}} w_\sigma e^{-S_\sigma}$$

\textbf{Key observation}: Each thimble $\sigma$ corresponds to a distinct 
\textbf{boundary fluid configuration}---a particular arrangement of the 
dissipative fluid at the stretched horizon.

\subsubsection{Thimbles as Physical States}

A Lefschetz thimble $J_\sigma$ attached to critical point $\phi_\sigma$ represents:

\textbf{Geometrically}: Steepest descent manifold in complexified configuration space

\textbf{Physically}: The boundary state $\phi_\sigma$ and all fluctuations around it that 
preserve the action's real part

\textbf{Dynamically}: A coherent boundary fluid configuration that minimizes 
dissipation while satisfying rotation constraints

The thimble is thus the \textbf{natural volume} around a boundary state---it 
includes all nearby configurations that belong to the same physical regime.

\subsubsection{Energy Stratification and Shells}

Different boundary states have different characteristic energies. The shell 
structure $\mathcal{S}_k$ emerges from organizing thimbles by their energy:

$$\mathcal{S}_k = \bigcup_{\sigma: E_\sigma \in [E_0\Lambda^k, E_0\Lambda^{k+1}]} J_\sigma$$

\textbf{Physical meaning}:
\begin{itemize}
\item Shell $\mathcal{S}_0$: Ground state boundary configurations (minimal energy)
\item Shell $\mathcal{S}_k$: Excited boundary states with energy $\sim E_0 \Lambda^k$
\item Shell transitions: Changes in boundary fluid regime
\end{itemize}

The Boltzmann suppression $e^{-\beta E_k}$ now has clear interpretation: 
\textbf{thermal probability of boundary being in energy regime $E_k$}.

\subsubsection{Why This Grounds the Formalism}

This physical interpretation resolves the question ``what are we integrating 
over?'' The path integral:

$$Z = \int \mathcal{D}\phi \, e^{-S[\phi]}$$

is summing over \textbf{all possible boundary fluid configurations}. The monodromy 
obstruction from rotation forces us to:

\begin{enumerate}
\item Complexify the configuration space (allow complex fluid states)
\item Decompose into thimbles (organize by boundary regime) 
\item Weight by action (favor low-dissipation configurations)
\end{enumerate}

\textbf{Result}: The dominant contributions come from \textbf{physical boundary states 
that minimize dissipation given rotation constraints}.

\subsubsection{Connection to 4D Timelines}

From the embedded boundary perspective, our observable 4D spacetime 
corresponds to:

\begin{itemize}
\item \textbf{One thermal sector}: Fixed temperature $\beta = \beta_{\text{obs}}$
\item \textbf{One dominant thimble}: The boundary configuration $J_0$ at that temperature
\item \textbf{One shell}: Typically $\mathcal{S}_0$ (ground state boundary)
\end{itemize}

Every boundary point on the stretched horizon defines a potentially valid Minkowski reference frame through local Wick rotation. There is no privileged "starting point" for spacetime—each point sees a different decomposition of past and future. Yet we observe a single, coherent classical reality. This is the convergence problem: how do multiple valid reference frames, each defining different possible timelines, yield a unique observable outcome?

The answer lies in the path integral structure. While all boundary configurations contribute as thimbles, the integral \textit{converges} onto configurations that minimize dissipation given rotation constraints. Subdominant thimbles persist geometrically (ensuring past states are not erased) but contribute negligibly to observables. The convergent nature of the path integral is thus not computational convenience but physical necessity—it is how the boundary fluid settles into a self-consistent configuration despite the multitude of potential starting points.

Other thimbles (other boundary configurations) remain as \textbf{geometric structure} 
even when not dominant. As temperature varies ($\beta$ changes), different thimbles 
become dominant---corresponding to \textbf{different effective timelines}.

This explains why ``past configurations persist'': they exist as \textbf{non-dominant 
boundary states} in the full thimble structure. Parameter control could 
shift dominance, making ``past'' configurations observable again.

\begin{remark}
The thimble formalism is thus not arbitrary mathematics but the \textbf{necessary 
structure for organizing dissipative boundary states} under rotation. The 
shells, convergence weights, and parameter-dependent accessibility all 
follow from the physics of boundary fluids in rotating geometries.
\end{remark}

\subsection{Radial Shell Decomposition}

\begin{definition}[Configuration Space Shell (Detailed)]
For shell spacing parameter $\Lambda > 1$ and ground state energy $E_0$, the $k$-th radial shell is:
\begin{equation}\label{eq:shell_def}
\mathcal{S}_k = \left\{\phi \in \mathcal{C} : E_{\text{dominant}}[\phi] \in [E_0\Lambda^k, E_0\Lambda^{k+1}]\right\}
\end{equation}
where $E_{\text{dominant}}[\phi] = \arg\max_n |\langle n | \phi \rangle|^2 \cdot E_n$ is the energy of the dominant mode.

Each shell $\mathcal{S}_k$ corresponds to a distinct relative homology class of Lefschetz thimbles in the complexified configuration space, attached to critical submanifolds with energy support in the range $[E_0\Lambda^k, E_0\Lambda^{k+1}]$.
\end{definition}

\begin{proposition}[Shell Partition]
The configuration space admits a decomposition over thimble classes:
\begin{equation}\label{eq:partition}
\mathcal{C}_{\text{total}} = \bigoplus_{k=0}^{\infty} \mathcal{S}_k
\end{equation}
where shells are distinguished by their energy labeling inherited from the KMS spectral decomposition.
\end{proposition}

\subsubsection{Hilbert Space Decomposition}

The field Hilbert space decomposes correspondingly:
\begin{equation}\label{eq:hilbert_decomp}
\mathcal{H}_{\text{total}} = \bigoplus_{k=0}^{\infty} \mathcal{H}_k
\end{equation}

where $\mathcal{H}_k$ contains states with dominant support in shell $\mathcal{S}_k$. Projector onto shell $k$:
\begin{equation}\label{eq:projector}
\hat{P}_k = \sum_{n: E_n \in [E_0\Lambda^k, E_0\Lambda^{k+1}]} |n\rangle\langle n|
\end{equation}

\subsection{Inter-Shell Dynamics}

\subsubsection{Effective Hamiltonian}

The full Hamiltonian in shell basis:
\begin{equation}\label{eq:H_shell}
H = \sum_{k} E_k \hat{P}_k + \sum_{k,k'} V_{k,k'} \hat{P}_k \hat{P}_{k'}
\end{equation}

where:
\begin{itemize}
\item $E_k$ is the characteristic energy of shell $k$
\item $V_{k,k'}$ is the inter-shell coupling induced by $\overline{V_I}$
\end{itemize}

\subsubsection{Coupling Strength}

From perturbation theory in $\overline{V_I}$:
\begin{equation}\label{eq:coupling}
V_{k,k'} \propto \langle k | \overline{V_I} | k' \rangle \sim \nu(E_k - E_{k'})
\end{equation}

This coupling is:
\begin{itemize}
\item \textbf{Weak} for adjacent shells ($k' = k \pm 1$): $V_{k,k\pm 1} \sim \nu E_0(\Lambda - 1)$
\item \textbf{Suppressed} for distant shells ($|k - k'| \gg 1$): $V_{k,k'} \sim e^{-|k-k'|}$
\end{itemize}

\subsection{Non-Exhaustion of Future Collapse Space}

A natural concern is whether convergence onto dominant thimbles exhausts all future measurement possibilities. We show this does not occur through three mechanisms:

\subsubsection{Irrational Phase Winding}

The shell amplitudes $\Lambda_n = e^{-\beta(1-i\nu)E_n}$ have angular parts $\theta_n = \beta\nu E_n$. For generic $\nu$ and spectra, the phases $\{\theta_n/2\pi\}$ are incommensurate, ensuring equidistribution mod $2\pi$ across higher shells. This prevents terminal phase-locking: the monodromy eigenvalues continue spiraling toward the origin rather than collapsing onto the positive real axis \cite{SignatureEmergence}. The system never settles into a single stationary configuration.

\subsubsection{Stokes Crossings and Parameter Drift}

Which thimbles contribute to the path integral is not fixed; it changes when control parameters cross Stokes surfaces \cite{BerryHowls1991}. In our framework, effective parameters include the rotation-induced phase (via $\nu$), the thermodynamic scale $\beta$, and the shell index through $E_k$. Slow drift or backreaction in these parameters produces intermittent re-entrance of dormant saddles: new channels can reappear even while the overall flow remains convergent. This is encoded through the time-dependent shell weights $c_k(\tau)$ and inter-shell couplings $V_{k,k'}$.

The standard Stokes condition for thimble transitions is:
\begin{equation}
\text{Im}(S_\alpha - S_\beta) \in \pi\mathbb{Z}
\end{equation}

In our shell language, this manifests as time-dependent coefficients $c_k(\tau)$ and occasional reweighting bursts when $\Delta S_k$ crosses critical values.

\subsubsection{Open-System Steady State}

Because Minkowski time serves as the dissipative outlet resolving the monodromy obstruction, we are never in a closed Euclidean system. This openness implies ongoing noise/drive (however small) sustaining nonzero flux into subdominant shells via $V_{k,k'}$. The result is a stationary distribution concentrated on dominant channels but maintaining permanent thin tails—hence future ``collapse space'' is never exhausted, though subdominant outcomes are rare.

This open-system character distinguishes our framework from closed quantum systems that would eventually thermalize completely.

\subsection{Decoherence Between Shells}

\subsubsection{General Mechanism}

In the open-system description with a stationary bath, standard decoherence theory \cite{Zurek2003, BreuerPetruccione} implies that off-diagonal density matrix elements between energy eigenstates decay due to dephasing. For our shell structure, this translates to suppression of coherences between shells $\mathcal{S}_k$ and $\mathcal{S}_{k'}$ with different characteristic energies.

The energy basis is a natural pointer basis for stationary environments since energy eigenstates are stationary under free dynamics and environment couplings typically commute (or nearly commute) with the Hamiltonian.

\subsubsection{Decoherence Rate Estimate}

Following standard einselection arguments, the decoherence rate between shells is approximately:
\begin{equation}\label{eq:decoherence}
\Gamma_{k,k'} \sim \gamma |E_k - E_{k'}|
\end{equation}

where $\gamma$ is an environment-dependent coupling strength. In our framework, $\gamma \sim \nu$ since the rotation-induced term provides the primary coupling to the dissipative Minkowski outlet.

For adjacent shells:
\begin{equation}
\Gamma_{k,k+1} \sim \nu E_0(\Lambda - 1)
\end{equation}

For widely separated shells:
\begin{equation}
\Gamma_{k,k'} \sim \nu E_0(\Lambda^{k'} - \Lambda^k) \quad \text{for } k' \gg k
\end{equation}

\subsubsection{Born Rule from Convergence Geometry}

The probability of observing configuration in shell $k$:
\begin{equation}
P_k = \frac{e^{-2\beta E_k}}{\sum_{k'} e^{-2\beta E_k'}}
\end{equation}

This reproduces Born rule when expressed in terms of wavefunction amplitudes, but the derivation is non-trivial. Standard quantum mechanics \textit{postulates} Born probabilities $P = |\langle \psi | n \rangle|^2$ as an axiom. We \textit{derive} them from convergence structure.

\textbf{The logic}:

\begin{enumerate}
\item Path integral includes all thimble contributions: $Z = \sum_\sigma w_\sigma e^{-S_\sigma}$
\item Thimble $J_k$ has action $S_k \approx \beta E_k$ (saddle-point approximation)
\item Weight includes phase: $w_k = e^{-\beta E_k} e^{i\theta_k}$ where $\theta_k = \beta \nu E_k$
\item Probability from squared amplitude: $P_k \propto |w_k|^2 = e^{-2\beta E_k}$
\item Born rule emerges: $|\langle n | \psi \rangle|^2 \propto e^{-2\beta E_n}$ for thermal state
\end{enumerate}

\textbf{Why this is non-trivial}:

While the Boltzmann factor $e^{-\beta E}$ is standard in thermal physics, its appearance here has different origin and meaning:

\textbf{Standard thermal physics}: Boltzmann weights count microstates at fixed temperature—a statistical statement.

\textbf{Our framework}: Weights measure \textit{geometric dominance} of thimbles in convergence hierarchy—a dynamical statement about which boundary configurations minimize dissipation.

The key insight: Born probabilities are not axioms about measurement or statistics about ensembles. They are \textit{geometric measures of convergence strength} in the path integral. The configuration that minimizes dissipation (maximizes $e^{-\beta E}$) dominates the convergence.

\textbf{Connection to measurement}: When observer at boundary point undergoes measurement (convergence event), the probability of observing outcome $k$ equals the geometric weight of thimble $J_k$ in the convergence. This is not "counting outcomes across worlds" (MWI) or "mysterious collapse" (Copenhagen)—it is \textit{dissipation-weighted geometric selection}.

\textbf{Testable consequence}: If Born probabilities come from convergence geometry, then factors affecting convergence (like controlling $\nu$ or $\beta$) should affect probability distributions in ways standard interpretations don't predict. This suggests experiments manipulating boundary dissipation to test probability shifts.

\section{Physical Consequences}
\subsection{Convergence vs Branching: The Structure of Ontological Plurality}

Both Many Worlds Interpretation and our framework accept ontological plurality—that multiple realities exist physically, not merely as descriptions. The crucial difference lies in how this plurality is structured and how it relates to measurement.

\subsubsection{MWI: Plurality Through Divergence}

Standard Many Worlds posits:
\begin{itemize}
\item \textbf{Process}: Measurement causes branching (one universe → many)
\item \textbf{Structure}: Branches are independent, causally disconnected
\item \textbf{Mechanism}: None—branching is postulated, not derived
\item \textbf{Observation}: "You" observe one outcome because you're in one branch
\end{itemize}

\subsubsection{Our Framework: Plurality Through Convergence Network}

Our framework shows:
\begin{itemize}
\item \textbf{Process}: Multiple boundary points define multiple convergences (many → constrained outcomes)
\item \textbf{Structure}: Convergences are interdependent, coupled through $V_{k,k'}$ and shared parameter space
\item \textbf{Mechanism}: Geometric—each convergence minimizes dissipation given rotation constraints
\item \textbf{Observation}: You observe one outcome because you're at one boundary point undergoing one convergence event
\end{itemize}

\subsubsection{The Network of Presents}

Every point on the stretched horizon defines a valid Minkowski reference frame and undergoes convergence. These convergences are not isolated—they form a \textit{self-consistent network} constrained by:

\begin{enumerate}
\item \textbf{Inter-shell coupling}: $V_{k,k'}$ couples configurations across shells
\item \textbf{Shared parameter space}: All convergences depend on same $(\beta, \nu, E_k)$
\item \textbf{Open-system dynamics}: Convergences influence each other through dissipative channels
\end{enumerate}

An observer at boundary point $x_i$ experiences convergence $C_i$ onto dominant thimble $J_i$. Another observer at $x_j$ experiences convergence $C_j$ onto $J_j$. These are distinct "present moments," yet they are \textit{mutually consistent} because the open-system coupling constrains which thimbles can simultaneously dominate.

\textbf{What you observe}: Not THE present, but A present—the convergence event at your boundary location.

\textbf{Why it feels unique}: Local convergence dominates your observation; other convergences are inaccessible from your boundary point without parameter navigation.

\textbf{Why it's not solipsistic}: Other convergences are ontologically real, and the coupling structure ensures they form compatible network (you can communicate with other observers, share history, etc.).

\subsubsection{Why This Explains Measurement}

MWI cannot explain \textit{why} you observe particular outcome—it only says "you're in a branch." Our framework provides geometric mechanism:

\textbf{You observe outcome $O_i$ because}:
\begin{enumerate}
\item Your boundary location undergoes convergence
\item Path integral at your location converges onto thimble $J_i$ that minimizes dissipation
\item Boltzmann and phase weighting: $w_i \propto e^{-\beta E_i}$ exponentially favors $J_i$
\item Decoherence suppresses other thimbles at rate $\Gamma \sim \nu|E_i - E_j|$
\end{enumerate}

This is not "you're in one universe"—it's "you're at one convergence point in a network of ontologically real but coupled convergences."

\subsubsection{Testable Distinction}

MWI predicts: Branches are causally isolated (no way to detect other branches)

Our framework predicts: Convergences are coupled (signatures of coupling detectable):
\begin{itemize}
\item Nested interference patterns from multiple shell contributions before convergence
\item Energy-dependent decoherence rates revealing coupling structure
\item Parameter-dependent shifts in dominant thimble (Stokes crossings)
\end{itemize}

The plurality is not in question—both frameworks have it. The question is: divergent branches (MWI) or convergent network (ours)?

\subsubsection{Shell Spacing from Boundary Physics}

To make the cosmological constant resolution quantitative, we must derive the shell spacing $\Lambda$ from boundary fluid properties rather than assuming it.

\textbf{Setup}: At the stretched horizon, boundary fluid has:
\begin{itemize}
\item Viscosity: $\nu$ (kinematic)
\item Rotation rate: $\Omega_H$ (horizon angular velocity)
\item Surface gravity: $\kappa = 1/(4M)$ for Schwarzschild, $\kappa = \kappa(a,M)$ for Kerr
\item Temperature: $T_H = \kappa/(2\pi)$, so $\beta = 2\pi/\kappa$
\end{itemize}

\textbf{Shell energy scale}: The ground shell $\mathcal{S}_0$ should have characteristic energy set by thermal scale:
$$E_0 \sim k_B T_H = \frac{\kappa}{2\pi}$$

For Schwarzschild black hole: $\kappa = 1/(4M)$, giving $E_0 \sim 1/(8\pi M)$.

\textbf{Shell spacing}: The ratio $\Lambda = E_{k+1}/E_k$ should relate to dissipation structure. From Navier-Stokes mapping, the imaginary potential scales as:
$$\bar{V}_I \sim \nu \Omega_H^2$$

The shell spacing emerges from how this dissipation organizes thimbles. Configurations separated by energy $\Delta E$ decohere at rate:
$$\Gamma \sim \nu \Delta E$$

For shells to be well-separated (decohere before inter-shell transitions), require:
$$\Gamma_{k,k+1} \cdot \tau_\text{thermal} \gtrsim 1$$
where $\tau_\text{thermal} = \beta$ is thermal timescale.

This gives:
$$\nu E_0 (\Lambda - 1) \cdot \beta \gtrsim 1$$

Solving for $\Lambda$:
$$\Lambda \gtrsim 1 + \frac{1}{\nu \beta E_0}$$

For Kerr black hole, $\nu$ relates to horizon properties. Near extremal rotation ($a \to M$), viscosity bound gives $\nu \sim \ell_\text{Planck}^2$. Using $\beta \sim M$ and $E_0 \sim 1/M$:
$$\Lambda \sim 1 + \ell_\text{Planck}^2 M^2 \sim 1 + (M/M_\text{Planck})^2$$

\textbf{Application to vacuum energy}:

For cosmological-scale structure, if observable universe corresponds to ground shell of structure at Hubble scale $R_H \sim 10^{28}$ cm:
$$\Lambda \sim (R_H/\ell_\text{Planck})^2 \sim 10^{122}$$

The ratio of modes in $\mathcal{S}_0$ to total:
$$\frac{N_{\mathcal{S}_0}}{N_\text{total}} \sim \frac{1}{\Lambda} \sim 10^{-122}$$

This gives vacuum energy suppression:
$$\frac{\rho_\text{obs}}{\rho_\text{naive}} \sim 10^{-122}$$

matching the observed discrepancy to within order of magnitude.

\textbf{Limitations}: This derivation assumes:
\begin{itemize}
\item Observable universe corresponds to black hole boundary at Hubble scale (speculative)
\item Shell structure scales from microscopic (Planck) to cosmological (Hubble)
\item Single dominant shell dominates vacuum energy
\end{itemize}

More rigorous derivation requires understanding cosmological boundary structure and how shell spacing scales across energy ranges. Nevertheless, this demonstrates that shell spacing $\Lambda \sim 10^{122}$ emerges naturally from boundary dissipation physics at cosmological scales, rather than being tuned by hand.

\textbf{Testable implication}: If shell spacing relates to dissipation, then environments with different effective viscosity should show different shell structures. Laboratory tests manipulating dissipation in quantum systems could probe this relationship at accessible scales.

\subsection{Double-Slit Interference from Multiple Shells}

\subsubsection{Setup}

Standard double-slit: electrons pass through slits separated by $d$, detected on screen at distance $L$.

\subsubsection{Nested Fringe Structure}

Each shell contributes interference with characteristic wavelength:
\begin{equation}\label{eq:wavelength_k}
\lambda_k = \frac{h}{p_k} = \frac{h}{\sqrt{2m E_k}}
\end{equation}

For shell $k$ with energy $E_k = E_0\Lambda^k$:
\begin{equation}
\lambda_k = \frac{\lambda_0}{\Lambda^{k/2}}
\end{equation}

Fringe spacing on screen:
\begin{equation}\label{eq:fringe_spacing}
\Delta y_k = \frac{\lambda_k L}{d} = \frac{\Delta y_0}{\Lambda^{k/2}}
\end{equation}

where $\Delta y_0$ is the fringe spacing from ground shell $\mathcal{S}_0$.

\subsubsection{Visibility}

The visibility of fringes from shell $k$:
\begin{equation}
\mathcal{V}_k = e^{-2\beta E_k} = e^{-2\beta E_0\Lambda^k}
\end{equation}

Higher shells produce finer fringes but with exponentially suppressed visibility.

\subsubsection{Total Pattern}

The observed intensity pattern is a superposition:
\begin{equation}\label{eq:total_intensity}
I(y) = \sum_{k=0}^{\infty} I_k \left[1 + \mathcal{V}_k \cos\left(\frac{2\pi y}{\Delta y_k}\right)\right]
\end{equation}

This produces nested interference: primary fringes from $\mathcal{S}_0$, secondary fine structure from $\mathcal{S}_1$, etc.

\subsection{Energy-Dependent Quantum Statistics}

\subsubsection{Boson vs Fermion Sectors}

Different shells can exhibit different quantum statistics if energy scale affects effective exchange symmetry.

For shell $k$:
\begin{equation}
|\psi_k\rangle_{\pm} = \frac{1}{\sqrt{2}}\left(|a\rangle|b\rangle \pm |b\rangle|a\rangle\right)
\end{equation}

where $\pm$ depends on shell-dependent phase factors.

\subsubsection{Observational Signature}

This predicts anomalous bunching/anti-bunching in multi-particle interferometry at specific energy scales corresponding to shell transitions.

\section{Structural Implications}

\subsection{Preservation of Past Configurations}

Unlike temporal slicing in standard formulations, where past field configurations are information-theoretically lost to decoherence, the shell structure embeds past states as persistent topological features. Each thimble represents not merely a saddle-point approximation but a stable homology class whose accessibility depends on control parameters rather than temporal distance.

This suggests a geometric distinction between:
\begin{itemize}
\item \textbf{Ergodic loss} (information scrambled but preserved in ensemble)
\item \textbf{Structural loss} (homology class becomes inaccessible)
\end{itemize}

The Stokes crossing dynamics indicate that ``inaccessibility'' is parameter-dependent rather than absolute. Configurations that contribute negligibly under one parametrization can become dominant under another—the system's effective ``memory'' is encoded in its full homological structure, not merely its current dominant contributions.

\subsection{Recursive Self-Reference}

The complexified configuration space, when organized via energy-labeled thimbles, exhibits a form of structural self-reference: the action $S$ that defines the thimbles depends on the same field configurations that flow along them. In the open-system steady state, this creates a self-consistent structure where:

\begin{enumerate}
\item The system's dynamics determine which configurations contribute
\item These contributing configurations feed back to define the dynamics
\item The steady state represents a fixed point of this recursion
\end{enumerate}

This recursive structure, combined with the non-exhaustion of collapse space, suggests that the geometry of possibility is not merely a mathematical artifact but encodes meaningful relationships between past, present, and potential configurations.

\subsection{Implications for Measurement and Observation}

\subsubsection{Multiple Convergences, Multiple Presents}

The boundary perspective reveals a fundamental feature: \textbf{convergence occurs at every boundary point}. Each point $x_i$ on the stretched horizon:
\begin{itemize}
\item Defines a valid Minkowski reference frame
\item Undergoes path integral convergence onto dominant configuration
\item Experiences this convergence as "measurement" or "present moment"
\end{itemize}

These convergences are ontologically real—not merely descriptive or perspectival. An observer at $x_i$ genuinely experiences convergence $C_i$, while an observer at $x_j$ genuinely experiences $C_j$. Both are actual events in the network structure.

\textbf{Key insight}: We do not observe THE present; we observe A present—the convergence happening at our boundary location. Other presents exist simultaneously at other boundary points, forming a self-consistent network through open-system coupling.

\subsubsection{Why Convergences Form Network, Not Isolated Worlds}

Unlike MWI's causally disconnected branches, our convergences are \textit{interdependent}:

\textbf{Coupling mechanisms}:
\begin{enumerate}
\item \textbf{Inter-shell coupling} $V_{k,k'}$: Configurations at different points coupled through shell interactions
\item \textbf{Shared parameters}: All convergences depend on same $(\beta, \nu, E_k)$—changing parameters affects network
\item \textbf{Stokes crossings}: When parameters drift, multiple convergence points simultaneously shift dominant thimbles
\item \textbf{Open-system steady state}: Dissipative channels couple convergences (one convergence affects others)
\end{enumerate}

\textbf{Physical manifestation}: If observer at $x_i$ could control parameters $(\beta, \nu)$, this would affect not only their own convergence but also convergences at other boundary points. The network is \textit{dynamically coupled}, not merely co-existing.

\subsubsection{Observable Consequences for Local Observer}

From your boundary point, you observe:
\begin{itemize}
\item \textbf{Single outcome}: Your local convergence onto dominant thimble
\item \textbf{Classical appearance}: Decoherence suppresses coherence between shells
\item \textbf{Born rule probabilities}: Thimble weights $e^{-\beta E_k}$ determine likelihood of outcomes
\item \textbf{Persistent past}: Previous convergences remain as subdominant thimbles in geometric structure
\end{itemize}

You do \textit{not} observe:
\begin{itemize}
\item Other boundary points' convergences (unless you navigate parameters)
\item Non-dominant thimbles at your location (suppressed by decoherence)
\item The full network structure (only your local node)
\end{itemize}

Yet the network structure has observable effects:
\begin{itemize}
\item Communication with other observers (they're at other network nodes)
\item Shared history (network consistency constraints)
\item Possibility of parameter-dependent navigation (Stokes crossings)
\end{itemize}

\subsubsection{Contrast with MWI}

\begin{center}
\begin{tabular}{|p{0.45\textwidth}|p{0.45\textwidth}|}
\hline
\textbf{Many Worlds} & \textbf{Convergence Network} \\
\hline
Plurality through divergence (one → many) & Plurality through multiple convergences (many → network) \\
\hline
Branches causally isolated & Convergences dynamically coupled \\
\hline
No mechanism for branching & Geometric mechanism: dissipation minimization \\
\hline
"You're in one branch" (no explanation) & "You're at one convergence point" (geometric selection) \\
\hline
Cannot detect other branches & Can detect coupling effects (interference, Stokes crossings) \\
\hline
Past unique, future branches & Past = previous convergences, future = possibility space \\
\hline
\end{tabular}
\end{center}

The fundamental difference: MWI has no explanation for \textit{why} you observe particular outcome (just "you're here"). Our framework: you observe outcome that minimizes dissipation at your boundary location, and this convergence is coupled to all other convergences in the network.

\subsubsection{Parameter Navigation and Network Access}

Because convergences form coupled network, parameter control potentially enables navigation:

If you change $(\beta, \nu, E_k)$ at your boundary point:
\begin{itemize}
\item Your dominant thimble may shift (Stokes crossing)
\item You access configuration that was subdominant before
\item This is "moving through network" not "traveling to other universe"
\end{itemize}

Subdominant thimbles at your location represent:
\begin{itemize}
\item Previous convergences (past states)
\item Convergences at other boundary points (other presents)
\item Potential future convergences (future possibilities)
\end{itemize}

The geometric framework does not forbid such navigation—thimble structure persists, accessibility is parameter-dependent. Whether practical control is achievable remains open.

\section{Connection to Existing Frameworks}

\subsection{Instantons and Tunneling}

Coleman-De Luccia instanton methods \cite{ColemanDeLuccia1980} provide a well-established framework for computing tunneling rates between vacuum states in quantum field theory. The approach identifies Euclidean solutions (instantons) whose non-trivial action encodes the tunneling amplitude between different vacuum configurations. These methods have been extensively applied to problems ranging from false vacuum decay to early universe cosmology.

Our shell structure shares mathematical similarities with instanton methods—both work in Euclidean signature, both identify critical configurations via action principles, and both compute transition amplitudes from action differences. However, the physical interpretation and scope differ significantly.

\textbf{Standard instanton picture}: Instantons describe rare tunneling events between distinct vacuum states within a single field space. The vacuum-to-vacuum transition is a dynamical process occurring at some spacetime point, with the instanton solution describing the field configuration during this transition. These are genuine quantum mechanical barrier penetration events, with the Euclidean continuation serving as a computational tool to extract WKB-type amplitudes.

\textbf{Our shell framework}: The shell structure is not about transitions between vacua within a single configuration space, but rather about the organization of configuration space itself under rotation-induced monodromy. Where instantons describe movement from one configuration to another, shells describe the stratification of the space of all configurations. Every configuration belongs to some shell; there is no "vacuum" and "excited state" dichotomy, only a hierarchy of boundary fluid regimes organized by energy.

The key distinction lies in universality and interpretation. Instantons are rare events—specific solutions that mediate specific transitions. Shell structure is universal—every field configuration in a rotating system organizes this way. Instantons happen at points in spacetime; shell decomposition is a property of the configuration space itself. Furthermore, in our picture, the Euclidean continuation is not merely computational. The signature transition from Euclidean to Minkowski is a physical necessity imposed by the monodromy obstruction, not an analytic trick.

That said, the technical machinery shares deep connections. The Picard-Lefschetz decomposition we employ to define shells is precisely the mathematical framework needed to make sense of instanton contributions when multiple saddle points contribute. Our thimbles play the same role as Lefschetz thimbles in instanton calculations—organizing the path integral around complex saddles. The difference is that we provide these thimbles with physical interpretation as boundary fluid configurations, rather than treating them purely as integration contours.

In summary: instanton methods describe what happens when the system moves between configurations. Shell structure describes how the space of configurations is organized. The former is dynamical (tunneling events); the latter is structural (geometric stratification). Both employ similar mathematical technology (Euclidean path integrals, complex saddles), but with different physical content and scope.

\subsection{Picard-Lefschetz Theory and Path Integral Convergence}

Recent work by Witten \cite{Witten2010}, Cristoforetti et al. \cite{Cristoforetti2012}, and others has established Picard-Lefschetz theory as the rigorous mathematical framework for defining oscillatory path integrals with complex weights. This approach has proven particularly powerful in lattice QCD, where the notorious "sign problem"—wild oscillations in the path integral due to complex fermion determinant—can be tamed by deforming the integration contour onto Lefschetz thimbles.

\subsubsection{The General Framework}

Consider a path integral with complex action:
$$Z = \int_{\mathcal{C}} \mathcal{D}\phi \, e^{-S[\phi]}$$

When $S$ has imaginary part (as occurs with rotation-induced $i\bar{V}_I$ in our case), the integrand oscillates wildly on the original real contour $\mathcal{C}$, making the integral ill-defined or numerically intractable. Picard-Lefschetz theory resolves this by:

\begin{enumerate}
\item \textbf{Complexifying} the configuration space: $\phi \in \mathbb{R}^n \to \phi \in \mathbb{C}^n$
\item \textbf{Identifying critical points}: Solutions to $\delta S/\delta \phi = 0$ (saddles in complex space)
\item \textbf{Defining thimbles}: For each saddle $\phi_\sigma$, the thimble $\mathcal{J}_\sigma$ is the steepest descent manifold—the locus where $\text{Re}(S)$ decreases most rapidly while $\text{Im}(S)$ remains constant
\item \textbf{Decomposing the integral}: The original contour is homologous to a sum over thimbles: $\mathcal{C} = \sum_\sigma n_\sigma \mathcal{J}_\sigma$ where $n_\sigma$ are intersection numbers
\end{enumerate}

The key advantage: on each thimble, the integrand is exponentially suppressed away from the saddle (by $\text{Re}(S)$) with no oscillations ($\text{Im}(S)$ constant). The path integral becomes:
$$Z = \sum_\sigma n_\sigma \int_{\mathcal{J}_\sigma} \mathcal{D}\phi \, e^{-S[\phi]}$$

Each thimble contribution can be evaluated via steepest descent, giving well-defined convergent integrals.

\subsubsection{Application to Rotating Systems}

In our framework, the monodromy obstruction from rotation introduces complex phases in the Euclidean evolution operator: $\Lambda_n = e^{-\beta(1-i\nu)E_n}$. This forces the path integral to be treated as oscillatory, requiring Picard-Lefschetz decomposition.

The thimbles $\mathcal{J}_\sigma$ in our case correspond to different boundary fluid configurations—each a distinct saddle of the action with dissipation from rotation. The intersection numbers $n_\sigma$ encode which thimbles contribute at given parameter values $(\beta, \nu, E_k)$, and change discontinuously at Stokes surfaces (Section 2.5.2).

\textbf{Witten's contribution}: Demonstrated that thimbles provide the correct analytic continuation from Euclidean to Lorentzian signature in gauge theories, resolving ambiguities in the path integral definition. His work showed that Stokes phenomena (discontinuous jumps in which thimbles contribute) have physical meaning—they encode phase transitions and changes in vacuum structure.

\textbf{Cristoforetti's contribution}: Applied thimble methods to lattice QCD at finite density, where complex fermion determinant causes severe sign problem. By deforming onto thimbles, they achieved convergent Monte Carlo sampling where standard methods fail completely. This demonstrated that thimbles are not just formal constructions but enable practical calculations.

\subsubsection{Our Contribution}

While Witten and Cristoforetti established the mathematical machinery and showed its utility for specific calculations, our contribution is threefold:

\textbf{Physical interpretation}: We identify thimbles with boundary fluid configurations in rotating geometries. They are not merely integration contours but represent distinct physical regimes of the boundary—different ways the dissipative fluid can minimize action while satisfying rotation constraints.

\textbf{Observable consequences}: We derive testable predictions from thimble structure (nested interference, energy-dependent decoherence) that distinguish this picture from alternatives. Standard Picard-Lefschetz applications focus on making integrals tractable; we focus on what the thimble structure tells us about physical reality.

\textbf{Convergence interpretation}: We show that the convergence of the path integral onto dominant thimbles is not just computational convenience but represents the physical process of measurement—the boundary selecting its configuration by minimizing dissipation. The thimble hierarchy encodes the network of possible "presents" (Section 4.3), with dominant thimbles corresponding to observed outcomes.

\subsubsection{Open Questions}

Several important questions remain:

\begin{itemize}
\item \textbf{Stokes crossing dynamics}: While we know Stokes jumps occur when $\text{Im}(S_\alpha - S_\beta) = n\pi$, the full dynamics of how the system evolves through these transitions requires better understanding. Do Stokes crossings correspond to observable events? Can they be controlled experimentally?

\item \textbf{Global vs local thimbles}: Our framework treats each boundary point as having its own thimble structure. How do these local decompositions relate to a global Picard-Lefschetz decomposition of the full field space?

\item \textbf{Numerical implementation}: Can the shell structure be computed numerically for realistic boundary fluid configurations? This would enable quantitative predictions beyond our qualitative analysis.
\end{itemize}

In summary: Picard-Lefschetz theory provides the rigorous mathematical foundation for our framework. Witten and Cristoforetti demonstrated its power for defining and computing path integrals. We provide physical interpretation—thimbles as boundary configurations, convergence as measurement, Stokes crossings as parameter-dependent accessibility—and connect this structure to observable phenomena.

\subsection{Stochastic Quantization}

Parisi-Wu formalism \cite{ParisiWu1981} introduces fictitious time $\tau$ with Langevin equation:
\begin{equation}
\frac{\partial\phi}{\partial\tau} = -\frac{\delta S}{\delta\phi} + \eta(\tau)
\end{equation}

\textbf{Our interpretation}: The Euclidean time $\tau$ from Wick rotation is literally this stochastic time, and the noise $\eta$ arises from coupling to hidden shells via $V_{k,k'}$.

\section{Discussion and Outlook}

\subsection{Summary of Results}

We have developed a radial shell decomposition of field configuration space based on the monodromy obstruction from rotational stress. Key findings:

\begin{enumerate}
\item Ontological plurality exists not through divergent branching (MWI) but through multiple simultaneous convergences forming a coupled network—each boundary point experiences a real "present" that is self-consistent with all others
\item Configuration space decomposes over energy-labeled Lefschetz thimbles $\mathcal{S}_k$ in the complexified space
\item Shells are topologically distinct (as relative homology classes) and exhibit controlled decoherence via standard open-system mechanisms
\item The shell structure provides physical content for quantum convergence and measurement
\item Observable vacuum energy comes from dominant shell $\mathcal{S}_0$, resolving cosmological constant discrepancy
\item Double-slit experiment produces nested interference from multiple shells, yielding testable predictions
\item Path integral consistency requires this geometric organization via KMS equilibrium, monodromy obstruction, and open-system dynamics
\item Convergence does not exhaust future collapse space due to irrational phase winding, Stokes crossings, and open-system steady state
\item Past configurations persist as accessible thimble structures, with parameter-dependent rather than absolute inaccessibility
\end{enumerate}

\subsection{Open Questions}

\subsubsection{Quantitative Cosmological Constant}

While the shell model provides qualitative resolution (suppression factor), precise prediction requires:
\begin{itemize}
\item Determining optimal shell spacing $\Lambda$
\item Identifying physical cutoff scale (Planck? String? Lower?)
\item Possible dynamical selection mechanism for observable shell
\end{itemize}

\subsubsection{Experimental Tests}

We propose three classes of experiments to test the convergent shell framework:

\textbf{1. Multi-scale electron interferometry}

\textit{Setup}: Double-slit experiment with electrons at variable energies, high-resolution position detection.

\textit{Prediction}: Nested interference patterns with fringe spacing $\Delta y_k = \Delta y_0/\Lambda^{k/2}$ and visibility $\mathcal{V}_k = \exp(-2\beta E_k)$. Specifically:
\begin{itemize}
\item Primary fringes from shell $\mathcal{S}_0$: spacing $\Delta y_0 \sim \lambda_0 L/d$ where $\lambda_0 = h/\sqrt{2mE_0}$
\item Secondary fine structure from $\mathcal{S}_1$: spacing $\Delta y_1 = \Delta y_0/\Lambda^{1/2}$, visibility suppressed by $e^{-2\beta E_0 \Lambda}$
\item Continue for higher shells with exponentially decreasing visibility
\end{itemize}

\textit{Control}: MWI predicts simple superposition (no nested structure). Copenhagen predicts single fringe pattern (no fine structure). Our framework uniquely predicts nested hierarchy.

\textit{Feasibility}: Electron microscopy routinely achieves sub-nanometer resolution. Detecting secondary fringes requires visibility $\mathcal{V}_1 \gtrsim 0.01$, achievable with $\beta E_0 \Lambda \lesssim 2$. For $E_0 \sim 1$ eV, $\Lambda \sim 10$, requires $\beta \sim 0.2$ eV$^{-1}$ (room temperature). \textbf{Immediately testable}.

\textit{Systematic errors}: Vibration, detector resolution, source coherence length. All controllable with modern equipment. Key requirement: high dynamic range detector (must resolve $\mathcal{V}_0 \sim 1$ and $\mathcal{V}_1 \sim 0.01$ simultaneously).

\textbf{2. Energy-dependent decoherence spectroscopy}

\textit{Setup}: Superconducting transmon qubit (multi-level system) at variable temperatures, measure $T_2$ (decoherence time) for different energy level transitions.

\textit{Prediction}: Decoherence rate between levels $n$ and $m$ scales as:
$$\Gamma_{n,m} = \nu |E_n - E_m|$$
For adjacent levels in shell $k$: $\Gamma_k \propto E_k$. Should see linear scaling of decoherence rate with energy.

\textit{Control}: Standard decoherence theory predicts various scalings (Ohmic: $\Gamma \propto \omega$; super-Ohmic: $\Gamma \propto \omega^3$; etc.) depending on bath. Our framework predicts specific form related to dissipation parameter $\nu$.

\textit{Feasibility}: Transmon qubits with 5-10 levels well-characterized. $T_2$ measurements routine. Temperature control enables varying $\beta$. \textbf{Testable within ~6 months in existing labs}.

\textit{Key signature}: Measure $\Gamma(E_n - E_m)$ for multiple transitions. Fit to $\Gamma = \nu |\Delta E| + \Gamma_0$ where $\Gamma_0$ is background. Extract $\nu$, check consistency across all transitions. Shell framework predicts universal $\nu$ independent of which levels measured.

\textbf{3. Weak measurement of shell contributions}

\textit{Setup}: Weak measurement protocol that partially reveals "which shell" without full decoherence. Based on weak values $\langle \psi_f | \hat{P}_k | \psi_i \rangle / \langle \psi_f | \psi_i \rangle$ where $\hat{P}_k$ projects onto shell $k$.

\textit{Prediction}: Before strong measurement (convergence), weak measurement should detect contributions from multiple shells with weights $e^{-\beta E_k}$. After strong measurement, only dominant shell contributes.

\textit{Protocol}:
\begin{enumerate}
\item Prepare state $|\psi_i\rangle$ with support across multiple shells
\item Weak measurement: couple shell projector $\hat{P}_k$ to meter with weak coupling $\epsilon \ll 1$
\item Measure meter, extract weak value (reveals shell $k$ contribution)
\item Repeat for different $k$, reconstruct shell distribution
\item Strong measurement: project onto energy eigenstate
\item Compare pre/post strong measurement shell distributions
\end{enumerate}

\textit{Feasibility}: Weak measurement established technique (Aharonov et al.). Applied successfully to photonic, atomic, solid-state systems. Requires: (1) tunable coupling to different energy levels (shell selector), (2) high-resolution meter. Both achievable. \textbf{Testable within 1-2 years}.

\textit{Distinctive prediction}: Our framework predicts \textit{convergence} onto dominant shell during strong measurement. Should see shell distribution sharpen from $P_k \propto e^{-\beta E_k}$ (pre-measurement) to $P_0 \approx 1$, $P_{k>0} \approx 0$ (post-measurement). MWI predicts branching (no convergence). Copenhagen offers no prediction for weak values during "collapse."

\textbf{Summary of experimental program}:

All three experiments leverage existing technology. Conservative timeline:
\begin{itemize}
\item Multi-scale interferometry: 6-12 months (most straightforward)
\item Decoherence spectroscopy: 6-18 months (requires systematic measurements)
\item Weak measurement protocol: 12-24 months (most technically demanding)
\end{itemize}

Together, these tests would:
\begin{itemize}
\item Confirm nested shell structure (test 1)
\item Verify energy-dependent convergence rates (test 2)
\item Directly observe convergence process (test 3)
\end{itemize}

Null results would falsify framework; positive results would distinguish from MWI/Copenhagen decisively.

\subsection{Philosophical Implications}

\subsubsection{Realism About Mathematical Structures}

The shell structure is forced by mathematical consistency (path integral convergence, monodromy obstruction, KMS conditions). This suggests that mathematical structures arising from consistency requirements have physical reality when indispensable for empirical predictions.

Higher shells are as real as electromagnetic fields or gravitational waves—we don't directly observe them, but they're necessary for explaining what we do observe.

\subsubsection{Relationalism About Observers}

Different observers couple to different shells depending on their measurement basis and control parameters. There is no absolute ``observable sector''—it is relational and parameter-dependent.

\textbf{Connection to quantum reference frames} \cite{Giacomini2019}: The shell decomposition might be observer-dependent (different reference frames access different shell structures).

\subsection{Future Directions}

\subsubsection{Non-Equilibrium Shell Dynamics}

This work assumed thermal equilibrium (Boltzmann weights $e^{-\beta E_k}$). Many physical systems of interest are far from equilibrium, raising questions about how shell structure behaves dynamically.

\textbf{Key questions}:
\begin{itemize}
\item How do shell populations $n_k(t)$ evolve under time-dependent driving?
\item Does shell-to-shell energy flow driven by $V_{k,k'}$ establish non-thermal steady states?
\item Can non-equilibrium conditions enhance accessibility of subdominant shells?
\end{itemize}

\textbf{Potential approach}: Extend the master equation formalism (Section 2.6) to include time-dependent transition rates between shells. The Lindblad equation for open quantum systems provides natural framework:
$$\frac{d\rho}{dt} = -i[H, \rho] + \sum_{k,k'} \gamma_{k,k'} \mathcal{L}_{k,k'}[\rho]$$
where $\mathcal{L}_{k,k'}$ are Lindblad operators encoding shell transitions.

\textbf{Physical motivation}: Many quantum systems (driven qubits, ultracold atoms, optomechanical systems) operate far from equilibrium. If shell structure persists in these regimes, it could be tested experimentally by driving systems between shells and measuring response.

\subsubsection{Curved Spacetime Generalization}

Our analysis assumed approximately flat boundary geometry (stretched horizon as 2D surface). Extending to curved spacetime raises fundamental questions:

\textbf{Cosmological backgrounds (FLRW)}: Does the apparent horizon in expanding universe have shell structure? The cosmological constant problem (Section 3.1) hints that cosmological scale might define a shell structure, but rigorous connection requires understanding boundary fluid in FLRW geometry.

\textbf{Black hole interiors (Kerr)}: We focused on exterior region near horizon. What is shell structure inside? Does it connect to inner horizon? The Penrose diagram suggests multiple boundary regions—how do their shell structures relate?

\textbf{Dynamical spacetimes}: During gravitational collapse or merger, spacetime geometry evolves. Does shell structure evolve correspondingly? Can shell transitions explain features of gravitational wave signals?

\textbf{Technical challenge}: In curved spacetime, defining "energy" is subtle (no timelike Killing vector generically). Shell structure might need to be defined locally at each boundary point, with global structure emerging from consistency constraints. The interplay between spacetime curvature and shell organization could reveal new physics.

\subsubsection{Quantum Information Perspective}

The shell structure naturally suggests quantum information theoretic questions:

\textbf{Entanglement entropy}: Shells $\mathcal{S}_k$ and $\mathcal{S}_{k'}$ are coupled through $V_{k,k'}$, suggesting entanglement. Computing entanglement entropy $S_{k,k'} = -\text{Tr}(\rho_k \log \rho_k)$ where $\rho_k$ is reduced density matrix for shell $k$ would quantify inter-shell correlations. Does this entropy relate to Bekenstein-Hawking entropy?

\textbf{Quantum complexity}: Transitions between shells involve navigating through parameter space (Section 4.3). The quantum complexity—minimum number of gates to prepare state in shell $k$ starting from shell $0$—might relate to classical action along thimble. This could connect shell structure to complexity geometry proposals in holography.

\textbf{Holographic correspondence}: In AdS/CFT, bulk geometry encodes boundary CFT. Do our shells (bulk boundary configurations) correspond to energy levels in dual CFT? The nested structure ($\mathcal{S}_0 \subset \mathcal{S}_1 \subset ...$) resembles Wilsonian RG flow. Exploring this connection might unify shell decomposition with holographic RG.

\textbf{Error correction}: Quantum error correction codes protect information by encoding in redundant subspace. Could shell structure provide natural error correction? Dominant shell contains observable information, subdominant shells maintain redundant copies. Parameter-dependent accessibility (Stokes crossings) might enable recovery of information from subdominant shells—a geometric form of quantum error correction.

\subsubsection{Laboratory Tests of Shell Structure}

While cosmological and black hole applications are important, laboratory tests would be more immediately accessible:

\textbf{Ultracold atoms}: Rotating Bose-Einstein condensates create vortices—analog of black hole frame-dragging. Can shell structure be seen in energy level structure of rotating condensate? Energy-dependent interference (Section 3.2) might be testable in matter-wave interferometry.

\textbf{Superconducting qubits}: Multi-level qubits (transmons) have energy spectrum. Applying rotating drive (time-dependent Hamiltonian with rotation) should induce monodromy. Measure: do higher levels show shell-like decoherence rates consistent with $\Gamma_{k,k'} \propto |E_k - E_{k'}|$?

\textbf{Optomechanics}: Rotating optical cavity coupled to mechanical oscillator. Photons carry angular momentum (rotation). Measure: energy-dependent decoherence of photon number states; nested interference patterns in photon statistics.

\textbf{Nuclear magnetic resonance}: Nuclear spins in rotating frame experience effective non-Hermitian evolution. Shell structure might manifest in relaxation rates of multi-spin states. Advantage: extremely precise control of parameters $(\beta, \nu)$ through temperature and rotation rate.

These systems share key feature: rotation + quantum coherence. Testing shell predictions in accessible systems would validate framework before applying to black holes or cosmology.

\section{Conclusions}
Rather than postulating causally isolated parallel worlds (MWI) or invoking mysterious wavefunction collapse (Copenhagen), we demonstrate that measurement is geometric convergence occurring at every boundary point simultaneously. Each convergence is ontologically real—yielding a genuine "present moment" at that location—yet all convergences form a self-consistent network coupled through open-system dynamics. You observe one outcome not because "you're in one branch" but because you're at one boundary location undergoing one convergence event. The multiplicity is not in question; the question is whether plurality arises through divergence (MWI) or through a network of coupled convergences (our framework). The Picard-Lefschetz decomposition provides the geometric structure for this convergent network, where every boundary point undergoes real measurement while maintaining consistency with all other measurements through inter-shell coupling and shared parameter space.

We have shown that the monodromy obstruction from signature emergence naturally induces a radial shell structure in field configuration space when combined with KMS equilibrium and open-system dynamics. This structure:

\begin{itemize}
\item Provides physical grounding for quantum measurement and branching (shells as distinct thimble homology classes)
\item Resolves cosmological constant problem (observable vacuum from single shell)
\item Makes testable predictions (multi-scale interference, energy-dependent decoherence)
\item Is necessary for path integral consistency (geometric organization required for well-defined Euclidean continuation)
\item Preserves past configurations as accessible topological structures with parameter-dependent accessibility
\item Exhibits recursive self-reference in its steady-state structure
\end{itemize}

The framework unifies disparate aspects of quantum theory—measurement, interference, vacuum energy, configuration space structure—under a single geometric principle: energy organization via Lefschetz thimbles in complexified configuration space, arising from the interplay of monodromy obstruction, thermal equilibrium, and open-system dynamics.

Rather than postulating parallel worlds or invoking mysterious collapse, we identify quantum convergence structure with pre-existing organization in quantum field theory: the Picard-Lefschetz decomposition into thimbles, labeled by energy via KMS conditions. This structure becomes physically manifest through the rotation-induced coupling $\overline{V_I}$ that breaks Euclidean periodicity and forces signature emergence.

The shell model demonstrates that foundational questions in quantum mechanics (what is the structure of configuration space? why do measurements yield definite outcomes? where is vacuum energy?) may have answers rooted in the geometric and topological properties of complexified field space—properties made visible through the lens of signature transitions and open-system dynamics.

\section*{Acknowledgments}

I thank the developers of Claude (Anthropic) for assistance in developing the mathematical framework and identifying connections to existing literature. I also thank Microsoft Copilot for detailed technical feedback on convergence mechanisms and rigorous grounding of the shell structure.

\begin{thebibliography}{99}

\bibitem{SignatureEmergence} A. Morgan, \textit{Signature Emergence from Rotational Stress: A Non-Wick Mechanism}, arXiv:XXXX.XXXXX (2025).

\bibitem{Zurek2003} W. H. Zurek, \textit{Decoherence, einselection, and the quantum origins of the classical}, Rev. Mod. Phys. \textbf{75}, 715 (2003).

\bibitem{BreuerPetruccione} H.-P. Breuer and F. Petruccione, \textit{The Theory of Open Quantum Systems} (Oxford University Press, 2002).

\bibitem{BerryHowls1991} M. V. Berry and C. J. Howls, \textit{Hyperasymptotics for integrals with saddles}, Proc. R. Soc. Lond. A \textbf{434}, 657 (1991).

\bibitem{ColemanDeLuccia1980} S. Coleman and F. De Luccia, \textit{Gravitational effects on and of vacuum decay}, Phys. Rev. D \textbf{21}, 3305 (1980).

\bibitem{Witten2010} E. Witten, \textit{Analytic continuation of Chern-Simons theory}, AMS/IP Stud. Adv. Math. \textbf{50}, 347 (2011), arXiv:1001.2933.

\bibitem{Cristoforetti2012} M. Cristoforetti et al., \textit{New approach to the sign problem in quantum field theories: High density QCD on a Lefschetz thimble}, Phys. Rev. D \textbf{86}, 074506 (2012), arXiv:1205.3996.

\bibitem{ParisiWu1981} G. Parisi and Y.-S. Wu, \textit{Perturbation theory without gauge fixing}, Sci. Sin. \textbf{24}, 483 (1981).

\bibitem{Giacomini2019} F. Giacomini, E. Castro-Ruiz, and Č. Brukner, \textit{Quantum mechanics and the covariance of physical laws in quantum reference frames}, Nat. Commun. \textbf{10}, 494 (2019), arXiv:1712.07207.

\end{thebibliography}

\end{document}
